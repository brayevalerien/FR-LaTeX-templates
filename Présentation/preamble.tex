%%%%%%%%%%%%%%%%%%%%%%%%%%%%%%%%%%%%%%%%%%%%%%%%%%%%%%%%%%%%%%%%%%%%%%%%%%%%%%%
%                              Paramètres de base                             %
%%%%%%%%%%%%%%%%%%%%%%%%%%%%%%%%%%%%%%%%%%%%%%%%%%%%%%%%%%%%%%%%%%%%%%%%%%%%%%%

% Certains packages sont déjà importés automatiquement par Beamer
% C'est le cas par exemple de xcolor, amsmath, amsthm, calc et hyperref

% Traduction de certains éléments en français + encodage valide des caractères spéciaux
\usepackage[french]{babel}
\usepackage[T1]{fontenc}
% Polices personnalisées
\usefonttheme{professionalfonts}
\usepackage{sourcesanspro}
\usepackage{sourcecodepro}
% Symboles mathématiques
\usepackage{amsmath, amssymb, eucal, bbold, bm}
% Formatage des symboles mathématiques
\usepackage[italic, eulergreek, symbolmisc]{mathastext}
% Permet une meilleure définition des commandes 
\usepackage{xparse}
% Personnalisation des légendes de graphiques
\usepackage{caption}


%%%%%%%%%%%%%%%%%%%%%%%%%%%%%%%%%%%%%%%%%%%%%%%%%
%                     Thème                     %
%%%%%%%%%%%%%%%%%%%%%%%%%%%%%%%%%%%%%%%%%%%%%%%%%

%%%%%%%%%%%%%%%%%%
%  Thème Beamer  %
%%%%%%%%%%%%%%%%%%

% Ce thème utilise comme base le thème Boadilla, chargé par les lignes suivantes
\usetheme{Boadilla}
\usecolortheme[]{seagull}

% Rend les éléments cachés invisibles (pour les animations)
\setbeamercovered{invisible}

% Suppression des symboles de navigation en pieds de page
\setbeamertemplate{navigation symbols}{}

% Taille des différents éléments
\setbeamerfont{title}{size=\fontsize{40}{35}, series=\scshape}
\setbeamerfont{author}{size=\large}
\setbeamerfont{frametitle}{series=\scshape}
\setbeamertemplate{frametitle}{\huge\MakeLowercase{\vspace{.25cm}\insertframetitle}}
\setbeamertemplate{frametitle continuation}{[\insertcontinuationcount]}
\setbeamerfont{itemize/enumerate subbody}{size=\normalsize}
\setbeamerfont{button}{size=\footnotesize}

%%%%%%%%%%%%%%
%  Couleurs  %
%%%%%%%%%%%%%%

% Les couleurs utilisées dans ce thème sont celles du thème Monokai Pro avec le filtre classique.
% Plus de couleurs sont disponibles à:
% https://gist.github.com/brayevalerien/cb94ac685ebc186f359deae113b6710c

\definecolor{background}{HTML}{272822}
\definecolor{button}{HTML}{414339}
\definecolor{textwhite}{HTML}{f8f8f2}
\definecolor{textgray}{HTML}{75715e}
\definecolor{textred}{HTML}{f92672}
\definecolor{textgreen}{HTML}{a6e22e}
\definecolor{textblue}{HTML}{66d9ef}
\definecolor{textmagenta}{HTML}{ae81ff}
\definecolor{textpurple}{HTML}{8c6bc8}

\setbeamercolor{background canvas}{bg=background}
\setbeamercolor{title}{fg=textmagenta}
\setbeamercolor{frametitle}{fg=textmagenta}
\setbeamercolor{normal text}{fg=textwhite}
\setbeamercolor{itemize item}{fg=textgray}
\setbeamercolor{itemize subitem}{fg=textgray}
\setbeamercolor{enumerate item}{fg=textgray}
\setbeamercolor{enumerate subitem}{fg=textgray}
\setbeamercolor{footline}{fg=textpurple}
\setbeamercolor{caption name}{fg=textwhite}
\setbeamercolor{button}{fg=textgray, bg=button}

%%%%%%%%%%%%%%%%%%%%%%
%  Numéro de slides  %
%%%%%%%%%%%%%%%%%%%%%%

% Ne conserve que: <n° slide actuelle>/<nombre de slides>
\setbeamertemplate{footline}[frame number]

%%%%%%%%%%%%%%%%%%%%
%  Slide de titre  %
%%%%%%%%%%%%%%%%%%%%

% Capitalisation du titre
\let\oldtitle\title
\renewcommand{\title}[1]{\oldtitle[]{\MakeLowercase{#1}}}

%%%%%%%%%%%%
%  Listes  %
%%%%%%%%%%%%

% Utilise des points pour les listes non ordonnées
\setbeamertemplate{itemize item}{$\bullet$}
\setbeamertemplate{itemize subitem}{$\circ$}

% Utilise des chiffres puis des lettres minuscules pour les listes ordonnées
\setbeamertemplate{enumerate item}{\arabic{enumi}.}
\setbeamertemplate{enumerate subitem}{\alph{enumii}.}


%%%%%%%%%%%%%%%%%%%%%%%%%%%%%%%%%%%%%%%%%%%%%%%%%%%%%%%%%%%%
%                     Commandes utiles                     %
%%%%%%%%%%%%%%%%%%%%%%%%%%%%%%%%%%%%%%%%%%%%%%%%%%%%%%%%%%%%

%%%%%%%%%%%%%%
%  Symboles  %
%%%%%%%%%%%%%%

\let\implies\Rightarrow  % ⟹
\let\impliedby\Leftarrow % ⟸
\let\iff\Leftrightarrow  % ⟺
\let\epsilon\varepsilon  % ε

%%%%%%%%%%%%%
%  Alertes  %
%%%%%%%%%%%%%

% Alerte standard (toujours rouge)
\NewDocumentCommand{\stdalert}{o g}{%
    \IfNoValueTF{#1}{\textcolor{textred}{#2}}%
    {\textcolor<#1>{textred}{#2}}
}

% Alerte verte
\NewDocumentCommand{\galert}{o g}{%
    \IfNoValueTF{#1}{\textcolor{textgreen}{#2}}%
    {\textcolor<#1>{textgreen}{#2}}
}

% Alerte rouge 
\NewDocumentCommand{\ralert}{o g}{%
    \IfNoValueTF{#1}{\textcolor{textred}{#2}}%
    {\textcolor<#1>{textred}{#2}}
}

% Alerte bleue
\NewDocumentCommand{\balert}{o g}{%
    \IfNoValueTF{#1}{\textcolor{textblue}{#2}}%
    {\textcolor<#1>{textblue}{#2}}
}

%%%%%%%%%%%%%
%  Figures  %
%%%%%%%%%%%%%

% Paramètres:
% 1. (Optionnel) Largeur de l'image. Défaut: 3/4 de la largeur du texte.
% 2. Chemin du fichier de l'image
% 3. Légende de l'image
\newcommand{\insertfig}[3][0.65]{
  \begin{figure}[H]
    \centering
    \includegraphics[width=#1\textwidth]{#2}
    \caption{#3}
  \end{figure}
}

% Image temporaire
\newcommand{\lipsumimage}[1][Lorem Ipsum]{
  \insertfig{./figures/lipsumimage}{#1}
}

%%%%%%%%%%%%%%%%%%%%%%%
%  Frames par défaut  %
%%%%%%%%%%%%%%%%%%%%%%%

% Frame de titre
% Les information apparaissant sur cette frame sont à définir au début de main.tex
\newcommand{\titleframe}{
    \begin{frame}
        \titlepage
    \end{frame}
}

% Frame de titre de partie
% Arguments:
% 1. Titre
% 2. Sous-titre
\newcommand{\chapterframe}[2]{
    \begin{frame}
        \huge\scshape\color{textmagenta}\MakeLowercase{#1}\\
        \normalsize\color{textwhite}#2
    \end{frame}
}

% Frame avec une liste non ordonnée
% Arguments:
% 1. Titre
% 2. Liste
\newcommand{\itemizeframe}[2]{
    \begin{frame}
        \frametitle{#1}
        \begin{itemize}
            #2
        \end{itemize}
    \end{frame}
}

% Frame avec une liste non ordonnée dont les éléments apparaissent progressivement 
% Arguments:
% 1. Titre
% 2. Liste
\newcommand{\itemizeframeanimated}[2]{
    \begin{frame}
        \frametitle{#1}
        \begin{itemize}[<+->]
            #2
        \end{itemize}
    \end{frame}
}

% Frame avec une liste ordonnée
% Arguments:
% 1. Titre
% 2. Liste
\newcommand{\enumerateframe}[2]{
    \begin{frame}
        \frametitle{#1}
        \begin{enumerate}
            #2
        \end{enumerate}
    \end{frame}
}

% Frame avec une liste ordonnée dont les éléments apparaissent progressivement 
% Arguments:
% 1. Titre
% 2. Liste
\newcommand{\enumerateframeanimated}[2]{
    \begin{frame}
        \frametitle{#1}
        \begin{enumerate}[<+->]
            #2
        \end{enumerate}
    \end{frame}
}

% Paramètres:
% 1. (Optionnel) Largeur de l'image. Défaut: 3/4 de la largeur du texte.
% 2. Titre de la frame
% 3. Chemin du fichier de l'image
% 4. Légende de l'image
\newcommand{\figframe}[4][.65]{
    \begin{frame}
        \frametitle{#2}
        \insertfig[#1]{#3}{#4}
    \end{frame}
}

% Dernière slide (avant d'éventuelles annexes)
% Argument:
% 1. Phrase de fin (remerciement, call-to-action...) (optionnelle)
\newcommand{\lastframe}[1][Merci pour votre attention.]{
    {
        \setbeamercolor{background canvas}{bg=textwhite}
        \begin{frame}
            \huge\color{background}#1
        \end{frame}
    }
    \appendix % tout ce qui suit n'apparaîtra pas dans la TOC
}