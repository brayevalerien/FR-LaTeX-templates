% Adapte le format et les marge
\usepackage[letterpaper, margin=3cm]{geometry}
% Ajoute le support d'UTF-8
\usepackage[utf8]{inputenc}
\usepackage[T1]{fontenc}
% Traduction en Français des mots clefs
\usepackage[french]{babel}


%%%%%%%%%%%%%%%%%%%%%%%%%%%%%
%   Formules de politesse   %
%%%%%%%%%%%%%%%%%%%%%%%%%%%%%

% Ajoute la formule d'ouverture de la lettre, ainsi que les coordonnées du destinataire et de l'expéditeur et la date.
% Par défaut "Madame, Monsieur"
\newcommand{\ouverture}[1][Madame, Monsieur]{
    \opening{#1,}
}

% Ajoute la formule de clôture de la lettre. Attention, n'inclut pas la signature.
% Par défaut "Cordialement"
% Remarque: closing n'est pas utilisée par manque de flexibilité, mais le code suivant s'en inspire.
\newcommand{\cloture}[1][Cordialement]{
    \par\nobreak\vspace{\parskip}%
    \stopbreaks
    \noindent
    \hspace*{4em}
    \parbox{.75\textwidth}{
        \bigskip
        \raggedright
        \ignorespaces #1, \\
        \bigskip
    }%
    \par
}

% Ajoute une signature. Requiert que \signature{} soit utilisée avant.
\newcommand{\signe}{
    \bigskip
    \hfill\fromsig
    \bigskip
}