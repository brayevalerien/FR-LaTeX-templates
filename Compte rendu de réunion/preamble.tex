%%%%%%%%%%%%%%%%%%%%%%%%%%%%%%%%%%%%%%%%%%%%%%%%%%%%%%%%%%%%%%%%%%%%%%%%%%%%%%%
%                               Packages de base                              %
%%%%%%%%%%%%%%%%%%%%%%%%%%%%%%%%%%%%%%%%%%%%%%%%%%%%%%%%%%%%%%%%%%%%%%%%%%%%%%%

% Réduit la taille des marges
\usepackage[a4paper, margin=2cm]{geometry}
% Ajoute plusieurs couleurs.
\usepackage[usenames,dvipsnames,pdftex]{xcolor}
% Permet de change le style des liens.
\usepackage{hyperref}
% Permet d'importer des images et des graphiques.
\usepackage{graphicx}
\graphicspath{{./}}
% Permet d'utiliser de nouvelles figures dans des environnements flottants.
\usepackage{float}
% Meilleure syntaxe pour les maths.
\usepackage{amsmath,amsfonts,mathtools,amsthm,amssymb}
% Permet de modifier le titre des figures.
\usepackage{caption}
% Traduction en Français des mots générés (date, table des matières...) et caractères français.
\usepackage[french]{babel}
\usepackage[T1]{fontenc}
% Ajout de structure conditionnels
\usepackage{xifthen}
% Entête pour chaque page
\usepackage{fancyhdr}

\def\class{article}

%%%%%%%%%%%%%%%%%%%%%%%%%%%%%%%%%%%%%%%%%%%%%%%%%%%%%%%%%%%%%%%%%%%%%%%%%%%%%%%
%                              Paramètres de base                             %
%%%%%%%%%%%%%%%%%%%%%%%%%%%%%%%%%%%%%%%%%%%%%%%%%%%%%%%%%%%%%%%%%%%%%%%%%%%%%%%

%%%%%%%%%%%%%%
%  Symboles  %
%%%%%%%%%%%%%%

\AtBeginDocument{\def\labelitemi{$\bullet$}} % change les tirets en points dans les listes
\AtBeginDocument{\def\labelitemii{$\circ$}}

%%%%%%%%%%%%%%%%%%%%%%%%%%%%%%%%%%
%  Modifie la couleur des liens  %
%%%%%%%%%%%%%%%%%%%%%%%%%%%%%%%%%%

\hypersetup{
  colorlinks,
  linkcolor=blue,
  urlcolor={blue!80!black}
}

%%%%%%%%%%%%%%%%%%%%%%%%%%%%%%%%%%%%%%%%%%%%%%%%%%%%%%%%%%%%%%%%%%%%%%%%%%%
%                     Commandes spécifiques à l'école                     %
%%%%%%%%%%%%%%%%%%%%%%%%%%%%%%%%%%%%%%%%%%%%%%%%%%%%%%%%%%%%%%%%%%%%%%%%%%%

%%%%%%%%%%%%%%%%%%%%%%%
% Titre avec fancyhdr %
%%%%%%%%%%%%%%%%%%%%%%%

\makeatletter


% Cette commande crée le titre du compte rendu.
% Paramètres:
% 1. (Optionnel) date du jour
% 2. Titre
% 3. (Optionnel) Domaine/projet
\newcommand\makecustomtitle[3][\today]{
  \fancypagestyle{plain}{
    \fancyhf{} % assure d'enlever tout ce qu'il y a dans l'entête avant de la personnaliser
    \ifthenelse{\isempty{#3}}{ % si le nom de projet a été fourni
      \fancyhead[R]{#1} % ajout de la date
    }{
      \fancyhead[L]{#3} % ajout du nom du projet
      \fancyhead[R]{#1} % ajout de la date
    }
  }
  \pagestyle{plain} % utilise l'entête pour toutes les pages
  \section*{\huge{#2}} % titre du rapport
}

\makeatother

%%%%%%%%%%%%%%%%%%%%%%%%%%%%%%%%%%%%%%%%%%%%%%%%%%%%%%%%%%%%%%%%%%%%%%%%%%%%%%%
%                                Environnements                               %
%%%%%%%%%%%%%%%%%%%%%%%%%%%%%%%%%%%%%%%%%%%%%%%%%%%%%%%%%%%%%%%%%%%%%%%%%%%%%%%

\usepackage{varwidth}
\usepackage{thmtools}
\usepackage[most,many,breakable]{tcolorbox}

\tcbuselibrary{theorems,skins,hooks}
\usetikzlibrary{arrows,calc,shadows.blur}

%%%%%%%%%%%%%%%%%%%%%%%%%%%%%%
%  Définitions des couleurs  %
%%%%%%%%%%%%%%%%%%%%%%%%%%%%%%

\definecolor{myblue}{RGB}{45, 111, 177}
\definecolor{mygreen}{RGB}{56, 140, 70}
\definecolor{example}{HTML}{2A7F7F}

%%%%%%%%%%%%%%%%%%%%%%
%  Commandes utiles  %
%%%%%%%%%%%%%%%%%%%%%%

% Paramètres:
% 1. Nom du théorème.
% 2. Color of theorem.
% 2. N'importe quels paramètres supplémentaires pour declaretheoremstyle.
% 3. N'importe quels paramètres supplémentaires pour mdframed.
% EXEMPLE:
% 1. \createnewcoloredtheoremstyle{thmdefinitionbox}{definition}{}{}
% 2. \createnewcoloredtheoremstyle{thmexamplebox}{example}{}{
%       rightline=true, leftline=true, topline=true, bottomline=true
%     }
% 3. \createnewcoloredtheoremstyle{thmproofbox}{proof}{qed=\qedsymbol}{backgroundcolor=white}
\newcommand\createnewcoloredtheoremstyle[4]{
  \declaretheoremstyle[
  headfont=\bfseries\sffamily\color{#2}, bodyfont=\normalfont, #3,
  mdframed={
    linewidth=2pt,
    rightline=false, leftline=true, topline=false, bottomline=false,
    linecolor=#2, backgroundcolor=#2!5, #4,
  },
  ]{#1}
}

%%%%%%%%%%%%%%%%%%%%%%%%%%%%%%%%%%%%%%
%  Crée les environnements de style  %
%%%%%%%%%%%%%%%%%%%%%%%%%%%%%%%%%%%%%%

\makeatletter

  % Environnements avec couleurs.

\createnewcoloredtheoremstyle{thmexamplebox}{example}{}{
  rightline=true, leftline=true, topline=true, bottomline=true
}

\makeatother

%%%%%%%%%%%%%%%%%%%%%%%%%%%%%
%  Crée les environnements  %
%%%%%%%%%%%%%%%%%%%%%%%%%%%%%

\declaretheorem[numbered=no, style=thmexamplebox, name=Participants]{participants}

\makeatletter

\newtcolorbox{todo}[1][]{%
  enhanced jigsaw,
  colback=gray!20!white,%
  colframe=gray!80!black,
  size=small,
  boxrule=1pt,
  title=\textbf{À faire avant la prochaine fois},
  halign title=flush center,
  coltitle=black,
  breakable,
  drop shadow=black!50!white,
  attach boxed title to top left={xshift=1cm,yshift=-\tcboxedtitleheight/2,yshifttext=-\tcboxedtitleheight/2},
  minipage boxed title=5.5cm,
  boxed title style={%
    colback=white,
    size=fbox,
    boxrule=1pt,
    boxsep=2pt,
    underlay={%
      \coordinate (dotA) at ($(interior.west) + (-0.5pt,0)$);
      \coordinate (dotB) at ($(interior.east) + (0.5pt,0)$);
      \begin{scope}
        \clip (interior.north west) rectangle ([xshift=3ex]interior.east);
        \filldraw [white, blur shadow={shadow opacity=60, shadow yshift=-.75ex}, rounded corners=2pt] (interior.north west) rectangle (interior.south east);
      \end{scope}
      \begin{scope}[gray!80!black]
        \fill (dotA) circle (2pt);
        \fill (dotB) circle (2pt);
      \end{scope}
    },
  },
  #1,
}

\newtcolorbox{Ordre du jour}{enhanced,
  breakable,
  colback=white,
  colframe=myblue!80!black,
  attach boxed title to top left={yshift*=-\tcboxedtitleheight},
  title=\textbf{Ordre du jour},
  boxed title size=title,
  boxed title style={
    sharp corners,
    rounded corners=northwest,
    colback=tcbcolframe,
    boxrule=0pt,
  },
  underlay boxed title={
    \path[fill=tcbcolframe] (title.south west)--(title.south east)
    to[out=0, in=180] ([xshift=5mm]title.east)--
    (title.center-|frame.east)
    [rounded corners=\kvtcb@arc] |-
    (frame.north) -| cycle;
  },
}

\NewDocumentEnvironment{ordredujour}{O{}O{}}
{\vspace{-10pt}\begin{Ordre du jour}{#1}{#2}}{\end{Ordre du jour}}


\newtcolorbox{Notes}{enhanced,
  breakable,
  colback=white,
  colframe=mygreen!80!black,
  attach boxed title to top left={yshift*=-\tcboxedtitleheight},
  title=\textbf{Notes},
  boxed title size=title,
  boxed title style={
    sharp corners,
    rounded corners=northwest,
    colback=tcbcolframe,
    boxrule=0pt,
  },
  underlay boxed title={
    \path[fill=tcbcolframe] (title.south west)--(title.south east)
    to[out=0, in=180] ([xshift=5mm]title.east)--
    (title.center-|frame.east)
    [rounded corners=\kvtcb@arc] |-
    (frame.north) -| cycle;
  },
}

\NewDocumentEnvironment{notes}{O{}O{}}
{\vspace{-10pt}\begin{Notes}{#1}{#2}}{\end{Notes}}

\makeatother

\newtheorem*{remark}{Remarque}